%%%%%%%%%%%%%%%%%%%%%%%%%%%%%%%%%%%%%%%%%%%%%%%%%%%%%%%%%%%%%%%%%%%%%%%%% 
% Autonomous Intelligent System, University of Bonn, LaTeX Beamer theme
% 
% Copyright (c) 2010-2013 Dirk Holz, dirk.holz@ieee.org
% 
% All rights reserved.
% 
% Redistribution and use in source and binary forms, with or without
% modification, are permitted provided that the following conditions
% are met:
% 
% 1. Redistributions of source code must retain the above copyright
% notice, this list of conditions and the following disclaimer.
% 2. Redistributions in binary form must reproduce the above copyright
% notice, this list of conditions and the following disclaimer in the
% documentation and/or other materials provided with the distribution.
% 
% THIS SOFTWARE IS PROVIDED BY THE COPYRIGHT HOLDERS AND
% CONTRIBUTORS ``AS IS'' AND ANY EXPRESS OR IMPLIED WARRANTIES,
% INCLUDING, BUT NOT LIMITED TO, THE IMPLIED WARRANTIES OF
% MERCHANTABILITY AND FITNESS FOR A PARTICULAR PURPOSE ARE
% DISCLAIMED. IN NO EVENT SHALL THE COPYRIGHT OWNER OR CONTRIBUTORS
% BE LIABLE FOR ANY DIRECT, INDIRECT, INCIDENTAL, SPECIAL,
% EXEMPLARY, OR CONSEQUENTIAL DAMAGES (INCLUDING, BUT NOT LIMITED
% TO, PROCUREMENT OF SUBSTITUTE GOODS OR SERVICES; LOSS OF USE,
% DATA, OR PROFITS; OR BUSINESS INTERRUPTION) HOWEVER CAUSED AND ON
% ANY THEORY OF LIABILITY, WHETHER IN CONTRACT, STRICT LIABILITY,
% OR TORT (INCLUDING NEGLIGENCE OR OTHERWISE) ARISING IN ANY WAY
% OUT OF THE USE OF THIS SOFTWARE, EVEN IF ADVISED OF THE
% POSSIBILITY OF SUCH DAMAGE.
% 
% The views and conclusions contained in the software and
% documentation are those of the authors and should not be
% interpreted as representing official policies, either expressed
% or implied, of the FreeBSD Project.

\documentclass[pdftex]{beamer}
\let\Tiny=\tiny

% all available options ---> 
% \usetheme[shadow,logoinnavbar,subsection,logoinframe,framenavbar]{AIS}
\usetheme[shadow,logoinnavbar,subsection]{AIS}

\usepackage{listings}

\title[The AIS Latex Beamer Class]{Presentations with the AIS Beamer Theme}
% \subtitle{}
\author[D. Holz]{\highlight{Dirk Holz}\inst{1}, Some Guy\inst{2}, Another Guy\inst{2,3}}
\institute[University of Bonn]
{
  \inst{1}%
  Department of Computer Science VI\\
  University of Bonn
  \and
  \inst{2}Some Institue
  \and
  \inst{3}Some other Institute}

\date{\today}

% This is only inserted into the PDF information catalog. Can be left
% out.
\subject{Talks}

% LOGO
% \pgfdeclareimage[height=0.5cm]{university-logo}{university-logo-filename}
% \logo{\pgfuseimage{university-logo}}
% \logo{\includegraphics[scale=0.18]{style/logo_uni_bonn_ais.pdf}}

% Delete this, if you do not want the table of contents to pop up at
% the beginning of each subsection:
% \AtBeginSubsection[]
\AtBeginSection[]
{
  \begin{frame}<beamer>
    \frametitle{Outline}
    \tableofcontents[currentsection,hideothersubsection]
  \end{frame}
}

% If you wish to uncover everything in a step-wise fashion, uncomment
% the following command:
% \beamerdefaultoverlayspecification{<+->}


\begin{document}

\begin{frame}
  \titlepage
\end{frame}

\begin{frame}
  \frametitle{Outline}
  \tableofcontents
  % \tableofcontents[pausesections]
\end{frame}


\section{Introduction}

\subsection{The Template Class}

\begin{frame}
  empty introduction
\end{frame}

\section{Contents}
\subsection[Typical Environments]{Typical Environments for Preseting Things}

\begin{frame}{Itemize, Enumerate and Description}
  \begin{itemize}
  \item one item
    \begin{itemize}
    \item first subitem
    \item second subitem
    \end{itemize}
  \item another item
  \item yet another item
  \end{itemize}
  \begin{enumerate}
  \item first item
  \item second item
  \item third item
  \end{enumerate}
  \begin{description}
  \item[ABCD] Characters
  \item[123432] Numbers
  \item[nothing] ``nothing'' in here
  \end{description}
\end{frame}


\begin{frame}{Blocks}
  \begin{block}{block with title}
    Content: Summarized information in block. 
  \end{block}  
  \begin{exampleblock}{An example}
    Some example in here!
  \end{exampleblock}
  \begin{alertblock}{An alert block}
    For something that's really important and/or a serious drawback of something.
  \end{alertblock}
  \begin{block}{}
    Content of a block without title 
  \end{block}  
\end{frame}


\section*{Summary}

\begin{frame}
  \frametitle<presentation>{Summary}

  \begin{itemize}
  \item The \alert{first main message} of your talk in one or two lines.
  \end{itemize}

  % The following outlook is optional.
  \vskip0pt plus.5fill
  \begin{itemize}
  \item Outlook
    \begin{itemize}
    \item Something you haven't solved.
    \item Something else you haven't solved.
    \end{itemize}
  \end{itemize}
\end{frame}

\end{document}
